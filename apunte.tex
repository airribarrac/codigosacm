\documentclass{book}
\usepackage{listings}
\usepackage[spanish]{babel}
\title{Apunte ICPC}
\begin{document}
{\let\cleardoublepage\clearpage 
	\maketitle	
	\tableofcontents

	\frontmatter
	\chapter{Notas previas}
	\section{Abreviaciones utilizadas}
\begin{lstlisting}[language=C]
typedef long long ll;
//en ciertos casos es necesario cambiar int por ll
typedef vector<int> vi;
typedef vector<vector<int> > vvi;
typedef pair<int,int> ii;
typedef vector<vector<ii> > vvii;		//util para grafos
typedef pair<pair<int,int>,int> iii;
#define mp(x,y) make_pair(x,y)
#define pb(x) push_back(x)
\end{lstlisting}

	\mainmatter
	\chapter{Estructuras de datos}
	\section{Fenwick Tree}
	\textbf{Nota:} Ambas implementaciones tienen rangos entre 1 a n.
	\subsection{Actualizaciones por rango, consultas puntales }
	\begin{lstlisting}[language=C]
struct FenwickTree{
  vi FT;
  FenwickTree(int N){
     FT.resize(N+1,0);
  }

  int query(int i){
     int ans = 0;
     for(;i;i-=i&(-i)) ans += FT[i];
     return ans;
  }

  int query(int i, int j){
     return query(j)-query(i-1);
  }

  void update(int i, int v){
     for(;i<FT.size();i+=i&(-i)) FT[i] += v;
  }

  void update(int i, int j, int v){
     update(i,v); update(j+1,-v);
  }
};

	\end{lstlisting}
	\pagebreak
	\subsection{Actualizaciones puntuales, consultas por rango}
	La consulta $query(a,b)$ corresponde a la sumatoria de los elementos entre los \'indices $a$ y $b$.
	\begin{lstlisting}[language=C]
struct FenwickTree {
  vi ft;
  FenwickTree(){}  
  FenwickTree(int n){
    ft.assign(n + 1, 0);
  }

  int query(int b) {
    int sum = 0;
    for (; b; b -= b&(-b)) sum += ft[b];
    return sum;
  }

  int query(int a, int b) {
    return query(b) - (a == 1 ? 0 : query(a - 1));
  }

  void update(int k, int v) {                    // note: n = ft.size() - 1
    for (; k < (int)ft.size(); k += k&(-k)) ft[k] += v;
  }
};
	\end{lstlisting}
	\section{Union-Find}
	\begin{lstlisting}[language=C]
	
	\end{lstlisting}
	
	
% para evitar doble pagina	
}
\end{document}